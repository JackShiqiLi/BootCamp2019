%!TeX spellcheck = en-US
% --------------------------------------------------------------
% This is all preamble stuff that you don't have to worry about.
% Head down to where it says "Start here"
% --------------------------------------------------------------


\documentclass[12pt]{article}

\usepackage[margin=1in]{geometry}
\usepackage{amsmath,amsthm,amssymb}
\usepackage{enumitem}
\usepackage{graphicx}
\usepackage{indentfirst}
\usepackage{tikz, tkz-euclide}
\usepackage{float}
\usepackage{changepage}
\usepackage{titling}
\usepackage{wrapfig}
\usetikzlibrary{calc}
\usetikzlibrary{positioning}
\graphicspath{ {./images/} }

\newcommand{\N}{\mathbb{N}}
\newcommand{\Z}{\mathbb{Z}}
\newcommand{\R}{\mathbb{R}}
\DeclareMathOperator{\arctantwo}{arctan2}


\setlength{\parindent}{10pt}
\interfootnotelinepenalty=10000
\setlength{\footskip}{15pt}


\newenvironment{theorem}[2][Theorem]{\begin{trivlist}
\item[\hskip \labelsep {\bfseries #1}\hskip \labelsep {\bfseries #2.}]}{\end{trivlist}}
\newenvironment{lemma}[2][Lemma]{\begin{trivlist}
\item[\hskip \labelsep {\bfseries #1}\hskip \labelsep {\bfseries #2.}]}{\end{trivlist}}
\newenvironment{exercise}[2][Exercise]{\begin{trivlist}
\item[\hskip \labelsep {\bfseries #1}\hskip \labelsep {\bfseries #2.}]}{\end{trivlist}}
\newenvironment{problem}[2][Problem]{\begin{trivlist}
\item[\hskip \labelsep {\bfseries #1}\hskip \labelsep {\bfseries #2.}]}{\end{trivlist}}
\newenvironment{question}[2][Question]{\begin{trivlist}
\item[\hskip \labelsep {\bfseries #1}\hskip \labelsep {\bfseries #2.}]}{\end{trivlist}}
\newenvironment{corollary}[2][Corollary]{\begin{trivlist}
\item[\hskip \labelsep {\bfseries #1}\hskip \labelsep {\bfseries #2.}]}{\end{trivlist}}

\begin{document}

% --------------------------------------------------------------
%                         Start here
% --------------------------------------------------------------

\setlength{\droptitle}{0pt}
\title{OSE2019 Week4 Firm Dynamics \\
Firm entry and exit}%replace X with the appropriate %number
\author{\vspace{0pt} Shiqi Li} %if necessary, replace with your %course title
\date{}

\maketitle

\textbf{Description of model} \par

    \begin{itemize}

        \item incumbent firm:
        \begin{itemize}
          \item production function: $y_{j t}=e^{\varepsilon_{j t}} k_{j t}^{\theta} n_{j t}^{\nu}$
          \item idiosyncratic productivity follows a Markov chain: $\varepsilon_{j t}$
          \item capital accumulation: $k_{j t+1}=(1-\delta) k_{j t}+i_{j t}$
          \item convex adjustment cost (units of output): $-\frac{\varphi}{2}\left(\frac{i_{j t}}{k_{j t}}\right)^{2} k_{j t}$
          \item fixed cost (units of output) to remain in operation each period: $c_{f}$
          \item if does not pay the fixed cost, the firm sells its capital with value: $(1-\delta) k$
        \end{itemize}

        \item potential entrants:
        \begin{itemize}
          \item continuum of potential entrants
          \item ex-ante identical
          \item at the beginning of each period, each firm decides whether to pay a fixed cost $c_e$ and enter the economy.
          \item if enters, then draw a value from idiosyncratic productivity $\varepsilon_{j t}$ from some distribution $\nu$ and begins an an incumbent firm with $k_{jt} = 0$.
          \item no adjustment costs at $k_{jt} = 0$.
          \item free entry among potential entrants, implies the expected value from entering is less than or equal to the entry cost $c_e$, so $c_{e} \leq \int v(\varepsilon, 0) \nu(d \varepsilon)$, with equality if entry actually takes place, i.e. $m^{*}>0$.
        \end{itemize}

      \item representative household:
      \begin{itemize}
        \item preference over consumption $C_t$ and labour supply $N_t$.
        \item expected utility function: $\mathbb{E}_{0} \sum_{t=0}^{\infty} \beta^{t}\left(\log C_{t}-a N_{t}\right)$
      \end{itemize}

      \item Steady state recursive competitive equilibrium:
      \begin{itemize}
        \item set of incumbent value functions $v(\varepsilon, k)$
        \item policy rules $k^{\prime}(\varepsilon, k)$ and $n(\varepsilon, k)$
        \item a mass of entrants per period $m^*$
        \item a measure of active firms at the beginning of the period $g^{*}(\varepsilon, k)$
        \item real wage $w^*$ such that :
        \begin{enumerate}
          \item incumbent firms maximize their firm value
          \item the free entry condition holds
          \item the labor market clears
          \item the measure of active firms $g^{*}(\varepsilon, k)$ is stationary
        \end{enumerate}
      \end{itemize}

      \item Calibration:
      \begin{align*}
        \theta &= 0.21 \\
        \nu &= 0.64 \\
        \delta &= 0.1 \\
        \beta &= 0.96 \\
        \varphi &= 0.5 \\
        \varepsilon_{j t+1} &= \rho \varepsilon_{j t}+\omega_{j t+1} \\
        \omega_{j t+1} &\sim N\left(0, \sigma^{2}\right) \\
        \rho &= 0.9 \\
        \sigma &= 0.02 \\
      \end{align*}
    \end{itemize}

\textbf{1: define the recursive competitive equilibrium}
  \begin{itemize}
    \item Firm optimization: taking $w^*$ as given, $v(\varepsilon, k)$ solves Bellman equation.
    $$v(\varepsilon, k)=\max \left\{(1-\delta) k, v^{1}(\varepsilon, k)-c_{f}\right\}$$
    $$v^{1}(\varepsilon, k)=\max _{k^{\prime}, n} \varepsilon k^{\theta} n^{\nu}-w^{*} n-\left(k^{\prime}-(1-\delta) k\right)-\frac{\varphi}{2}\left(\frac{k^{\prime}}{k}-(1-\delta)\right)^{2} k+\beta \mathbb{E}\left[v\left(\varepsilon^{\prime}, k^{\prime}\right)\right]$$
    Solve for policy rules $k'(\varepsilon, k)$ and $n(\varepsilon, k)$
    \item Free entry condition holds with market clearing:
    $$c_{e} \leq \int v(\varepsilon, 0) \nu(d \varepsilon)$$
    Solve for $w^*$ together with firm's optimization Problem
    \item Solve for stationary measure of incumbent firms $g^*(\varepsilon, k)$
  \end{itemize}

\textbf{2: solve representative firm steady state equilibrium}\\
Suppose that there are no entry and exits, all firms parcitipate in the market in all periods, and $k' = k = \bar{k}$, $\varepsilon = \varepsilon' = 0$,  then we have
  \begin{itemize}

    \item Firm's FOC:
    $$-1 - \varphi \delta + \beta V_k(k') = 0$$

    \item Firm's envelope:
    \begin{align*}
      V_{k}(k)
      &= \pi_{k}(k)+(1-\delta) - \frac{\varphi}{2}\left[-2 \frac{i}{k}+\left(\frac{i}{k}\right)^{2}\right] \\
      &= \pi_{k}(k)+(1-\delta)- \frac{\varphi}{2}\left\{\left[\frac{k^{\prime}(k)}{k}-(1-\delta)\right]^{2}-2\left[\frac{k^{\prime}(k)}{k}-(1-\delta)\right]\right\}
      \intertext{Hence in steady state,}
      \tilde{V}_{k}\left(k^{*}\right)
      &= \pi_{k}\left(k^{*}\right)+(1-\delta)-\frac{\varphi}{2}\left(\delta^{2}-2 \delta\right)\\
      &= \pi_{k}\left(k^{*}\right)-\frac{\varphi}{2} \delta^{2}+(\varphi-1) \delta+1
    \end{align*}

    \item By firm's profit maximization:
    \begin{align*}
      \pi(k) &= \max _{n} e^{\epsilon} k^{\theta} n^{\nu}-w n
      \intertext{take derivative w.r.t n,}
      \implies \nu e^{\epsilon} k^{\theta} n^{\nu-1}-w &= 0\\
      \implies w &= \nu e^{\epsilon} k^{\theta} n^{\nu-1}
    \end{align*}

    \item Combining everying above to get Euler equation:
    \begin{align*}
      -1-\varphi \delta+\beta\left[\theta \bar{k}^{\theta-1} n^\nu-\frac{\varphi}{2} \delta^{2}+(\varphi-1) \delta+1\right]=0
    \end{align*}

  \end{itemize}


% --------------------------------------------------------------
%     You don't have to mess with anything below this line.
% --------------------------------------------------------------

\end{document}
